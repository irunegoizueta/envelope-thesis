%preámbulo de latex para apuntes
% añado paquetes interesantes
\usepackage[titles]{tocloft} %para reformatear los ´indices
\usepackage[latin1]{inputenc}  % caracteres con acentos
\usepackage{latexsym}  % para símbolos
\usepackage{amsmath} %para tener más fórmulas matemáticas
\usepackage{amssymb} %más símbolos
\usepackage{mathrsfs} %más símbolos
\usepackage{amsfonts} %más símbolos
\usepackage[pdftex]{graphicx}  % para poder incluir gráficos
%\usepackage[pdftex, draft]{graphicx}  % para poder incluir gráficos, pero no los muestra
%\usepackage{subfigure} %para poder incluir subfiguras
\usepackage[all] {xy}
\usepackage{bookman}   % cambio a un font más atractivo
%\usepackage{mathptmx}  % font times, que es el mostrado por word
\usepackage{color}  %para poder usar colores
%\usepackage{pstricks} % para poder pintar diagramas
\usepackage{url} %para utilizar el comando url que permite poner direcciones webs
%\usepackage[spanish]{babel} %para poner en castellano los títulos automáticos
\usepackage{marvosym} % Para el euro
\usepackage{titlesec}  % para poder cambiar la apariencia de los títulos
\usepackage{makeidx}  %para hacer un índice alfabético
\makeindex %habilita los comandos del índice
\usepackage{fancyhdr}  % para poder definir unas cabeceras y pies de página bonitos 
\usepackage{pdfpages} %para poder incluir pdfs completos
\usepackage{wallpaper} %para poner una portada bonita y fondos en cualquier p'agina
\usepackage{nomencl}
\usepackage{capt-of}  % permite poner captions fuera de elementos flotantes.
%\usepackage{tocloft} 
%\makeglossary
%\usepackage[nodisplayskipstretch]{setspace}
%\setstretch{1.5}
\usepackage[latin1]{inputenc}
\usepackage{tikz}
\usetikzlibrary{shapes,arrows}
\usepackage{longtable}
\usepackage{tabu}
\usepackage[autostyle=true]{csquotes}
%\setdefaultlanguage{english}
\usepackage{amsmath}
\usepackage{algorithm}
\usepackage[noend]{algpseudocode}

\makeatletter
\def\BState{\State\hskip-\ALG@thistlm}
\makeatother

% modificaci´on del formato del ´indice
\setlength{\cftchapnumwidth}{1.9em} %para evitar el solape de los números de cap´itulossecciones por encima de 100 en el índice
\setlength{\cftsecnumwidth}{2.8em} %para evitar el solape de los números de secciones por encima de 100 en el índice
\setlength{\cftsubsecnumwidth}{4em} %para evitar el solape de los números de subsecciones por encima de 100 en el índice
\setlength{\cftfignumwidth}{3em} %para evitar el solape de los números de figuras por encima de 100 en el índice
\setlength{\cfttabnumwidth}{3em} %para evitar el solape de los números de tablas por encima de 100 en el índice

%\usepackage{glossary/gloss}     % para añadir un glosario de términos. Comentado por que no está en el paquete básico de latex


%dimensiones y márgenes de la página. Se pueden usar como unidades mm en lugar de in
 \setlength\topmargin{0in} %Margin at top of page above all printing. Add 1 inch (so that, for example, setting \topmargin to 0.25in would produce a top margin of 1.25 inches).
\setlength\headheight{14pt}  % 	Height of the header (the header is text that appears atop all pages).

\setlength\headsep{0.25in}  %Distance from bottom of header to the body of text on a page.
\setlength\textheight{9in} %Height and width of main text box
\setlength\textwidth{6.5in}
\setlength\oddsidemargin{0in} %Left margin on odd numbered pages. Add 1 inch (as with \topmargin).
\setlength\evensidemargin{0in} %Left margin on even numbered pages. Add 1 inch (as with \topmargin).
\setlength\parindent{0.25in} % Amount of indentation at the first line of a paragraph.
\setlength\parskip{0.125in}  % 	Distance between paragraphs.
\setlength\lineskip{0.125in}  % Distance between lines.

%diseño de cabeceras y pies de página
\pagestyle{fancyplain} %páginas con encabezado y pie

\fancyhf{} % borrado de todos los ajustes previos

\renewcommand{\chaptermark}[1]{\markboth{\thechapter. #1}{}} %configuración de las cabecerar
\renewcommand{\sectionmark}[1]{\markright{\thesection. #1}}


%\******[\fancyplain{página 1&página impar}{página impar}]    {\fancyplain{1&página par}{página par}} 

\lhead[\fancyplain{}{\thepage}]       {\fancyplain{}{\rightmark}} 

\chead[\fancyplain{}{}]       {\fancyplain{}{}}

\rhead[\fancyplain{}{\leftmark}]       {\fancyplain{}{\thepage}}

\lfoot[\fancyplain{}{}]       {\fancyplain{}{}}

\cfoot[\fancyplain{\thepage}{}]       {\fancyplain{\thepage}{}}

\rfoot[\fancyplain{} {}]      {\fancyplain{}{}}

\renewcommand{\headrulewidth}{0 pt} %modifica el ancho de las líneas de cabecera 
\renewcommand{\footrulewidth}{0 pt} % el pie sin línea

%formateo de la apariencia de los títulos de capítulo

\newcommand{\bigrule}{\titlerule[0.5mm]}

\titleformat{\chapter}[display] % cambiamos el formato de los capítulos
{\bfseries\Huge} % por defecto se usarán caracteres de tamaño \Huge en negrita
{% contenido de la etiqueta
 %\titlerule % línea horizontal
 \filleft % texto alineado a la derecha
 \Large\chaptertitlename\ % "Capítulo" o "Apéndice" en tamaño \Large en lugar de \Huge
 \Large\thechapter} % número de capítulo en tamaño \Large
{0mm} % espacio mínimo entre etiqueta y cuerpo
{\filleft} % texto del cuerpo alineado a la derecha
[\vspace{0.5mm} \bigrule] % después del cuerpo, dejar espacio vertical y trazar línea horizontal gruesa

%configuración para poder tener una lista de notación
%\newcommand{\listofnotation}{\input CPR_symbols.tex \clearpage}
%\newcommand{\addnotation} [3]{$ #1 $ \> \parbox{6.25in}{#2 \dotfill  \pageref{#3}}\\}
%\newcommand{\newnot}[1]{\label{#1}}


%otras configuraciones interesantes
\sloppy  %suaviza las reglas de ruptura de líneas de Latex
\frenchspacing %usar espaciado normal después de punto

%enumeración de los ejercicios
\newcounter{exercise}[chapter]
\newcommand{\exerciseNumber}{Exercise \stepcounter{exercise} \thechapter.\theexercise}
\newcommand{\refExerciseNumber}{exercise \thechapter.\theexercise}

\newcounter{solExercise}[chapter]
\newcommand{\solExerciseNumber}{Solution to exercise \stepcounter{solExercise} \thechapter.\thesolExercise}
\newcommand{\refSolExerciseNumber}{Solution to exercise \thechapter.\thesolExercise}

%poniendo bonitas las tablas

\usepackage{booktabs} %para cambiar el grosor de l�neas
%\toprule[0.08em]
%\midrule[0.05em]
%\bottomrule[0.08em]
\usepackage{array} %para poder alinear verticalmente en tablas
%p{5cm} %top
%b{5cm} %bottom
%m{5cm} %middle
